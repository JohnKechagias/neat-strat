\documentclass[12pt]{article}
\usepackage[a4paper, total={6in, 10in}]{geometry}
\begin{document}

\section*{Abstract}

An experiment was conducted where neural networks are trained throught the use of the
NEAT algorithm and self-play to learn a deterministic game resembling checkers. Firstly,
we look at an overview of the NEAT algorithm, explain the game rules and look at the 
way the models were trained. In greater detail, multilayer feedfoward neural networks 
were used to evaluate alternative bord positions and make moves based on the principal
variation search algorithm.

\section*{Introduction}

Neuroevolution of augmented topologies (NEAT) is an evolutionary algorithm used to create
neural networks.

\section*{Implementation Specification}

An implementation of the NEAT algorithm, written in Python, was created for this paper.

\section*{Game Rules}

The game is played on a five by five board with hexadecimal tiles (see Fig 1). There are
two players, denoted as \textbf{red} and \textbf{blue}. Players start on opposite
corners of the board with one piece each. Each board tile can hold a maximum of ten
pieces. We say that a player owns a tile when he has pieces on it. The blue player moves
first and then play alternates between sides. The goal of the game is to eliminate all
of the opponent's pieces. There are two moves available to each player, \textbf{transfer}
and \textbf{production}.

\subsection*{Transfer Move}

Players can select any number of their owned pieces from a tile and transfer them to a 
neighbouring tile. For this move to be valid, the following conditions must be met:
\begin{itemize}
  \item The starting tile and the destination tile are adjacent.
  \item The number of selected pieces must not exceed the total number of pieces on the
    starting tile.
  \item The total number of pieces on the destination tile must not exceed the maximum 
    allowed number of pieces on a tile.
\end{itemize}
If the destination tile is occupied by the opposing player, a fight ensues. During the
fight, the number of tiles possessed by the player with the greater quantity is 
subtracted from both sides.

\subsection*{Production Move}

Players can select a tile they occupy and add one more piece to it. For this move to be
valid, the number of pieces on the tile must be less than the maximum allowed number of 
pieces on a tile. For example:

\end{document}
